\documentclass[a4paper,14pt]{article}
\usepackage[english,russian]{babel}	% локализация и переносы
%\setlength\parindent{1.25}
\usepackage{fontspec}
	\setmainfont{Times New Roman}
\usepackage[14pt]{extsizes}	

\usepackage{setspace}
	\singlespacing
\usepackage{graphicx} % Required for including pictures
\usepackage{fancyhdr}
	\pagestyle{fancy}
	\fancyhf{}
	\renewcommand{\headrulewidth}{0mm}

\usepackage[utf8]{inputenc}

\usepackage{ragged2e}
\usepackage{microtype}
\tolerance=500
\hyphenpenalty=10000
\emergencystretch=3em		
\nopagebreak


%% Свои команды
\newcommand{\mysection}[1]{\section*{#1} \addcontentsline{toc}{subsection}{#1}}
\newcommand{\mysubsection}[1]{\subsection*{#1} \addcontentsline{toc}{subsubsection}{#1}}

%%% Страница
\usepackage{geometry} % Простой способ задавать поля
\geometry{top=15mm}
\geometry{bottom=30mm}
\geometry{left=30mm}
\geometry{right=20mm}

\renewcommand{\labelenumii}{\arabic{enumi}.\arabic{enumii}}
\renewcommand{\labelenumiii}{\arabic{enumi}.\arabic{enumii}.\arabic{enumiii}}
\begin{document}
	
		
	\vspace{20.8pt}
	
	\begin{flushright}
		\begin{minipage}{0.4\textwidth}
			\textbf{	ІНФОРМАЦІЙНА}
		\end{minipage}
	\end{flushright}
	
	\begin{flushright}
		\begin{minipage}{0.35\textwidth}
			Начальнику ГУ ДПС в області\\
			Ірині СТОЛЯРИК
		\end{minipage}
	\end{flushright}
		\vspace{20.8pt}
	
	\begin{center}
		\textbf{	ДОПОВІДНА ЗАПИСКА}
	\end{center}	
	
	\vspace{20.8pt}
	
	
	\textbf{ Від:} Управління по роботі з податковим боргом ГУ ДПС в області\\
		
				
	\textbf{ Тема:} на виконання п.п. 4.1 та 4.2 протоколу апаратної наради ГУ ДПС в області №01-п від 16.06.2024\\
	
	\vspace{10.8pt}

	
	
	\textbf{ Суть питання:} Управлінням по роботі з податковим боргом проведено інвентаризацію вжитих заходів по єдиному соціальному внеску та сформовано план - графік заслуховування крупних боржників.\\
	

	\setlength{\itemindent}{1.75em} Станом на 01.02.2024 року кількість боржників по єдиному соціальному внеску складала 7833, а сума їх боргу становила 164,9 млн.гривень.
    Структура боргу має наступний вигляд:
    \begin{itemize}
        \item станом на 21.02.2024 борг погашено  - по 161 боржнику на суму 911,4 тис. грн.;
        \item борг банкрутів - 6924 тис.грн. по 23 боржниках;
        \item борг державних підприємств - 21706 тис.грн. по 27 боржниках;
        \item борг комунальних підприємств - 5957,3 тис.грн. по 51 боржнику;
        \item борг юридичних осіб (не банкрути, комунальні, державні) - 31946,8 тис. грн. по 981 боржнику;
        \item борг фізичнх осіб - 97469,3 тис.грн. по 6593 боржниках, в т.ч. зі станом 11,12 - по 31023,3 тис.грн. по  3895 боржниках.
    
    \end{itemize}

 На виконанні в ДВС та ДКСУ знаходяться виконавчі документи по 6730 боржниках на суму боргу 160374,0 тис.грн., в т.ч.:
 \begin{itemize}
     \item по економічно активних боржниках - на суму 39854,1 тис.грн. по 1615 боржниках;
     \item по боржниках із відсутніми ознаками економічної активності - суму 84954,3 тис.грн. по 4968 боржниках;
 \end{itemize}

 Не на виконанні в ДВС заборгованість по 942 боржниках на суму 3614 тис.грн. - новостворена заборгованість по 28 боржниках на суму 274,5 тис. грн. та заборгованість, на суму якої не сформовані вимоги з технічних причин.

 В 2024 році (станом на 21.02.2024) в рахунок погашення заборгованості ЄСВ надійшло 3456,1 тис.грн. по 249 боржниках, в т.ч. найбільші суми погашено такими боржниками:
 \begin{itemize}
     \item ТОВ «СОЛІД-УКРАЇНА» - 686,6 тис.грн.;
     \item ПРАТ «ІВАНО-ФРАНКІВСЬКТОРФ» - 250 тис.грн.;
     \item КП «ГАЛИЧВОДОКАНАЛ» ГАЛИЦЬКОЇ МІСЬКОЇ РАДИ - 151 тис.грн.;
     \item ЛЕНИК МИХАЙЛО СТЕПАНОВИЧ - 117 тис.грн.;
     \item КНП «НОВИЦЬКИЙ ЦЕНТР ПМСД» - 110 тис.грн.
 \end{itemize}

Сформовано план - графік заслуховування крупних боржників (29 боржників на суму боргу 24,4 млн.гривень, додається).

  
	\vspace{10.8pt}
 

	
	
	\vspace{20.8pt}
	\begin{minipage}{0.4\textwidth}
	Начальник управління по роботі з податковим боргом ГУ ДПС в Івано-Франківській області
	\end{minipage}
	\hspace{5ex} \hspace{1ex} \hspace{5ex} {\normalsize Світлана ТОМАШІВСЬКА }
	
	\vspace{15.0pt}	
	\hspace{1ex}Погоджено:
	\vspace{5pt}
		
	\vspace{20.8pt}
	
	\begin{minipage}{0.4\textwidth}
		Заступник начальника ГУ ДПС в Івано-Франківській області
	\end{minipage}
	\hspace{8ex} \hspace{1ex} \hspace{5ex} {\normalsize Наталія ЗІКРАТА }

	
	\medskip 
	
	\begin{flushleft}
		\hspace{5ex}{\footnotesize 	Вик. Олександр Фролов IP2194}
	\end{flushleft}  
	
\end{document}	
	
	